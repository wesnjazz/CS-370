\documentclass[10pt,letterpaper]{article}
\usepackage[top=1in,bottom=1in,left=1in,right=1in]{geometry}
\usepackage{datetime}
\usepackage{natbib}      % http://merkel.zoneo.net/Latex/natbib.php
\usepackage{palatino}
\usepackage{verbatim}
\usepackage[normalem]{ulem}
\bibpunct{(}{)}{;}{a}{,}{,}

\usepackage{array}

\usepackage{chngpage}
\usepackage{stmaryrd}
\usepackage{amssymb}
\usepackage{amsmath}
\usepackage{graphicx}
\usepackage{lscape}
\usepackage{subfigure}
\usepackage[usenames,dvipsnames]{color}
\definecolor{myblue}{rgb}{0,0.1,0.6}
\definecolor{mygreen}{rgb}{0,0.3,0.1}
\usepackage[colorlinks=true,linkcolor=black,citecolor=mygreen,urlcolor=myblue]{hyperref}

\newcommand{\bocomment}[1]{\textcolor{Bittersweet}{BO says: #1}}

\newcommand{\ignore}[1]{}
\newcommand{\transpose}{^\mathsf{T}}
\newcommand{\inner}[1]{\langle #1 \rangle} 
\newcommand{\smallsec}[1]{\noindent \textbf{#1\ }}
\newcommand{\cmd}[1] {{\color{blue}\texttt{#1}}}

\newcommand{\solution}[1]{{\color{myblue} \emph{[Solution:} 

#1 

\emph{End solution]}}}
\newcommand{\solutionnote}[1]{{\color{myblue} \emph{[Note:}

#1 

\emph{End note]}}}
\newcommand{\points}[1]{{\color{mygreen}\emph{[#1]\ \ }}}

\newcommand{\aone}{\diamondsuit}
\newcommand{\atwo}{\heartsuit}
\newcommand{\bone}{\triangle}
\newcommand{\btwo}{\Box}
\newcommand{\myand}{\ \land\ }
\newcommand{\myor}{\ \lor\ }
\newcommand{\mynot}{\lnot}

\title{
  Homework 1 solution template\\
  \Large{CMPSCI 370 Spring 2019, UMass Amherst} \\
  \Large{Name: Subhransu Maji} \\
}


\settimeformat{ampmtime}
\date{}
\begin{document}
\maketitle

\renewcommand\thesubsection{\thesection.\alph{subsection}}

Here is a template that your solutions should roughly follow. Include outputs as figures, and code should be included in the end.

\section{Matrix manipulation}
\begin{enumerate}
\item Create a $m \times n$ array of all zeros.
\vspace{1in}
\item Create a random $m \times n$ array.
\vspace{1in}
\item Code to compute its length (or norm).
\vspace{1in}

\item  Given variables $u$ and $v$ representing arrays of size $n \times 1$, write down code to compute their
\begin{enumerate}
\item dot product
\vspace{1in}
\item angle
\vspace{1in}
\item distance
\vspace{1in}
\end{enumerate}
\item Given an array $a \in \mathbb{R}^{m\times n}$ write code to reshape it to a vector of size $nm \times 1$.
\vspace{1in}
\end{enumerate}


\section{Image formation}

\begin{enumerate}
\item Illustration of the object and the image formed in the pinhole camera. 
\vspace{2in}
\item Calculations for the size of the object.
\vspace{1in}
\item Calculations for the distance.
\vspace{1in}
\item Time taken for the camera with a lens.
\vspace{1in}
\end{enumerate}

\newpage
\section{Aligning Prokudin-Gorskii images} 
\begin{enumerate}
\item Outputs of \cmd{evalAlignment} on the toy images.
\item Output of \cmd{alignChannels.m} that shows the computed shifts as seen below.

\begin{verbatim}
>> alignProkudin
 1 00125v.jpg shift: G ( 0, 0) B ( 0, 0)
 2 00153v.jpg shift: G ( 0, 0) B ( 0, 0)
 3 00398v.jpg shift: G ( 0, 0) B ( 0, 0)
 4 00149v.jpg shift: G ( 0, 0) B ( 0, 0)
 5 00351v.jpg shift: G ( 0, 0) B ( 0, 0)
 6 01112v.jpg shift: G ( 0, 0) B ( 0, 0)
\end{verbatim}

\item A figure that shows all the aligned color images. Only include the images from the Prokudin-Gorskii dataset in the original resolution. For example, do not take low-resolution screenshots of the outputs; Instead save them using appropriate commands in Matlab and Python.

\begin{figure}[h]
\begin{tabular}{ccc}
\includegraphics[width=0.32\linewidth]{../output/prokudin-gorskii/00125-aligned.jpg} & 
\includegraphics[width=0.32\linewidth]{../output/prokudin-gorskii/00149-aligned.jpg} & 
\includegraphics[width=0.32\linewidth]{../output/prokudin-gorskii/00153-aligned.jpg} \\
\includegraphics[width=0.32\linewidth]{../output/prokudin-gorskii/00351-aligned.jpg} & 
\includegraphics[width=0.32\linewidth]{../output/prokudin-gorskii/00398-aligned.jpg} & 
\includegraphics[width=0.32\linewidth]{../output/prokudin-gorskii/01112-aligned.jpg} \\
\end{tabular}
\caption{\label{fig:aligned} Aligned color images.}
\end{figure}
\end{enumerate}

\newpage
\section{Color image demosaicing}
\begin{enumerate}
\item Errors of the nearest neighbor interpolation algorithm. 

\begin{verbatim}
>> evalDemosaicing
--------------------------------------------
# 	 image             baseline   	nn
--------------------------------------------
1 	 balloon.jpeg      0.179239 	 0.179239 
2 	 cat.jpg           0.099966 	 0.099966 
3 	 ip.jpg            0.231587 	 0.231587 
4 	 puppy.jpg         0.094093 	 0.013670 
5 	 pencils.jpg       0.181449 	 0.181449 
6 	 candy.jpeg        0.206359 	 0.206359 
7 	 house.png         0.117667 	 0.117667 
8 	 light.png         0.097868 	 0.097868 
9 	 sails.png         0.074946 	 0.074946 
10 	 tree.jpeg        0.167812 	 0.167812 
--------------------------------------------
 	 average            0.139150 	 0.139150
--------------------------------------------
\end{verbatim}

\item A figure (e.g., Figure~\ref{fig:demosaic-output}) that shows the images obtained after interpolation.

\begin{figure}[h!]
\begin{tabular}{cccc}
\includegraphics[width=0.225\linewidth]{../output/demosaic/balloon-nn-dmsc.jpg} & 
\includegraphics[width=0.225\linewidth]{../output/demosaic/pencil-nn-dmsc.jpg} & 
\includegraphics[width=0.225\linewidth]{../output/demosaic/ca-nn-dmsc.jpg} & 
\includegraphics[width=0.225\linewidth]{../output/demosaic/i-nn-dmsc.jpg} \\
\includegraphics[width=0.225\linewidth]{../output/demosaic/ligh-nn-dmsc.jpg} & 
\includegraphics[width=0.225\linewidth]{../output/demosaic/squirre-nn-dmsc.jpg} & 
\includegraphics[width=0.225\linewidth]{../output/demosaic/hous-nn-dmsc.jpg} &
\includegraphics[width=0.225\linewidth]{../output/demosaic/sail-nn-dmsc.jpg} \\
\includegraphics[width=0.225\linewidth]{../output/demosaic/pupp-nn-dmsc.jpg} & 
\includegraphics[width=0.225\linewidth]{../output/demosaic/tree-nn-dmsc.jpg} \\
\end{tabular}
\caption{\label{fig:demosaic-output} Demosaiced images using nearest neighbor interpolation.}
\end{figure}

\item The three plots for the \cmd{puppy.jpg} image.
\begin{figure}[h!]
\begin{tabular}{cccc}
\includegraphics[width=0.33\textwidth]{plot_red.jpg} & 
\includegraphics[width=0.33\textwidth]{plot_green.jpg} & 
\includegraphics[width=0.33\textwidth]{plot_blue.jpg} \\
(a) red & (b) green & (c) blue
\end{tabular}

\caption{\label{fig:plot} The value of pixels against the values of their neighbor.}
\end{figure}
\end{enumerate}

\section{Solution code}
Include the source code for your solutions as seen below (only the files you implemented are necessary). 
In latex the command \cmd{verbatiminput\{alignChannels.m\}} allows you to include the code verbatim as seen below. 
Regardless of how you do this the main requirement is that the included code is readable (use proper formatting, variable names, etc.)
A screenshot of your code works too provided you include a link to source files.



\subsubsection{alignChannels.m}
\verbatiminput{../initial/alignChannels.m}
\subsubsection{mosaicImage.m}
\verbatiminput{../initial/mosaicImage.m}
\subsubsection{demosaicImage.m}
\verbatiminput{../initial/demosaicImage.m}
\end{document}
