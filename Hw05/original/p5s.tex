\documentclass[10pt,letterpaper]{article}
\usepackage[top=1in,bottom=1in,left=1in,right=1in]{geometry}
\usepackage{datetime}
\usepackage{natbib}      % http://merkel.zoneo.net/Latex/natbib.php
\usepackage{palatino}
\usepackage{verbatim}
\usepackage[normalem]{ulem}
\bibpunct{(}{)}{;}{a}{,}{,}
\usepackage[draft]{graphicx}

\usepackage{enumitem}
\usepackage{array}

\usepackage{chngpage}
\usepackage{stmaryrd}
\usepackage{amssymb}
\usepackage{amsmath}
\usepackage{graphicx}
\usepackage{lscape}
\usepackage{subfigure}
\usepackage[usenames,dvipsnames]{color}
\definecolor{myblue}{rgb}{0,0.1,0.6}
\definecolor{mygreen}{rgb}{0,0.3,0.1}
\usepackage[colorlinks=true,linkcolor=black,citecolor=mygreen,urlcolor=myblue]{hyperref}

\newcommand{\bocomment}[1]{\textcolor{Bittersweet}{BO says: #1}}

\newcommand{\ignore}[1]{}
\newcommand{\transpose}{^\mathsf{T}}
\newcommand{\inner}[1]{\langle #1 \rangle} 
\newcommand{\smallsec}[1]{\noindent \textbf{#1\ }}
\newcommand{\cmd}[1] {{\color{blue}\texttt{#1}}}

\newcommand{\solution}[1]{{\color{myblue} \emph{[Solution:} 

#1 

\emph{End solution]}}}
\newcommand{\solutionnote}[1]{{\color{myblue} \emph{[Note:}

#1 

\emph{End note]}}}
\newcommand{\points}[1]{{\color{mygreen}\emph{[#1]\ \ }}}

\newcommand{\aone}{\diamondsuit}
\newcommand{\atwo}{\heartsuit}
\newcommand{\bone}{\triangle}
\newcommand{\btwo}{\Box}
\newcommand{\myand}{\ \land\ }
\newcommand{\myor}{\ \lor\ }
\newcommand{\mynot}{\lnot}

\title{
  Homework 5 solution template\\
  \Large{CMPSCI 370 Spring 2019, UMass Amherst} \\
  \Large{Name: Subhransu Maji} \\
}


\settimeformat{ampmtime}
\date{}
\begin{document}
\maketitle

\renewcommand\thesubsection{\thesection.\alph{subsection}}

Here is a template that your solutions should roughly follow. Include outputs as figures, and code should be included in the end.


\section{Decision trees}
\begin{enumerate}
\item Empty decision tree. 
 
\begin{itemize}
\item accuracy on training set: \underline{\hspace{2cm}}
\item accuracy on test set: \underline{\hspace{2cm}}
\item explain your answer:
\end{itemize}
\vspace{1.in}

\item Decision tree of depth 1
\begin{enumerate}
\item The coordinate of pixel with the highest accuracy: $(x, y)=$\underline{\hspace{2cm}}. Figure~\ref{fig:viz} visualizes the scores.
\begin{figure}[h]
\centering
\includegraphics[draft, width=0.4\linewidth]{../code/score.jpg}
%comment the previous line and uncomment the next line to show your figure
%\includegraphics[width=0.4\textwidth]{../code/score.jpg}
\caption{\label{fig:viz}Visualizing scores}
\end{figure}
\item Write down the decision tree that obtains the best classification
\vspace{2in}
\item Accuracy of decision tree with depth 1 on test set: \underline{\hspace{2cm}}
\vspace{0.3in}
\end{enumerate}

\item Decision tree of depth 2.
\begin{enumerate}
\item Write down the decision tree in if-then-else statement
\vspace{4in}
\item Accuracy of decision tree with depth 2 on test set: \underline{\hspace{2cm}}
\end{enumerate}

\end{enumerate}
\newpage

\section{Linear classifier}
\begin{itemize}
\item Accuracy of linear classifier on test set: \underline{\hspace{2cm}}
\item Figure ~\ref{fig:weights} visualizes positive and negative parts of weights.
\begin{figure}[h]
\centering
\begin{tabular}{cc}
\includegraphics[draft, width=0.4\linewidth]{../code/positive_weights.jpg} &
\includegraphics[draft, width=0.4\linewidth]{../code/negative_weights.jpg} \\
%comment the previous two lines and uncomment the next two lines to show your figure
%\includegraphics[width=0.4\linewidth]{../code/positive_weights.jpg} &
%\includegraphics[width=0.4\linewidth]{../code/negative_weights.jpg} \\
(a) positive weights & (b) negative weights
\end{tabular}
\caption{\label{fig:weights}Visualizing weights}
\end{figure}
\end{itemize}

\section{Nearest neighbor classifier}
Figure~\ref{fig:knn} plots the test accuracy against the number of $k$ for k nearest-neighbor classifier

\begin{figure}[h]
\centering
\includegraphics[draft, width=0.4\linewidth]{../code/knn.jpg}
%comment the previous line and uncomment the next line to show your figure
%\includegraphics[width=0.4\textwidth]{../code/knn.jpg}
\caption{\label{fig:knn}Accuracy of k-nearest-neighbor classifier}
\end{figure}

\section{Bag-of-visual-words representations}
\begin{enumerate}
\item Visualize the dictionary
\begin{figure}[h]
\centering
\includegraphics[draft, width=0.5\linewidth]{dictionary.pdf}
\end{figure}
\item \textbf{Extra credit}: test accuracy: \underline{\hspace{2cm}}
\end{enumerate}
\section{Solution code}
Include the source code for your solutions as seen below (only the files you implemented are necessary). 
In latex the command \cmd{verbatiminput\{alignChannels.m\}} allows you to include the code verbatim as seen below. 
Regardless of how you do this the main requirement is that the included code is readable (use proper formatting, variable names, etc.)
A screenshot of your code works to provided you include a link to source files.



\subsection{scoreFeatures.m}
\verbatiminput{../code/scoreFeatures}
\subsection{code training a decision tree}
\subsection{code evaluate a decision tree on test set}
\subsection{code for training and evaluating the linear model}
\subsection{code for k-nearest neighbor classification}
\subsection{constructDictionary.m}
\subsection{encodeImage.m}
\end{document}
